%%
%%   TEMPLATE of kaishi
%%   コピペ用
%%

%% typeset: latexmk (uplatex -> dvipdfmx)
\documentclass[uplatex,dvipdfmx]{vkaishi}

% \usepackage{v-hyperref}
\usepackage{vuccaken}
\usepackage{template} % Your sty File

% \analogtrue
% \colorfalse

%% スタイルファイルの読み込み(\usepackage{})や自作マクロは,
%% プリアンブル(ここ)ではなく,自分のスタイルファイル(.sty)に書く.

\begin{document}

%% - - - - - - - - - - - - - - - - - %%
\title{会誌原稿テンプレート}% タイトル
\author[テンプレ]{太郎}% 名前
\belong{テンプレ科学科}{2}% 所属・回生
%% - - - - - - - - - - - - - - - - - %%

\mokuji{2} % 目次出力
\maketitle % タイトル出力

%% - - - - - - - - - 以下本文 - - - - - - - - - - - %%


%% - - - - - - - - - - - - - - - - - - - - - - - -
\section*{はじめに}
%% - - - - - - - - - - - - - - - - - - - - - - - -
はじめに,背景・導入・目的などを書くとよい.\par
概要を書いてもよい.


%% - - - - - - - - - - - - - - - - - - - - - - - -
\section{セクション}
%% - - - - - - - - - - - - - - - - - - - - - - - -
chapterは使わない\footnote{タイトル出力がchapterに相当します}.\par
section以下は自由に使ってよし.


%% - - - - - - - - - - - - - - - - - - - - - - - -
%%  参考文献
%% - - - - - - - - - - - - - - - - - - - - - - - -
%% \bibitem のlabelはユニークでなければいけないので,各自変更すること.
%% 著者名+出版年でラベリングするのがおすすめです.
\begin{thebibliography}{99}
  \bibitem{label-1} 著者1・著者2,『本のタイトル』,出版社,出版年.
  \bibitem{label-2} ページの著者,『ページのタイトル』,最終アクセス日,\\(\url{https://vuccaken.github.io}).
\end{thebibliography}

\end{document} % - - - - - - - - - - - - - - - - - - - - -
%%
%% ファイトだよ!
%%