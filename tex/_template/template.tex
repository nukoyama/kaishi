%%
%%   TEMPLATE of kaishi
%%   コピペ用
%%

%% typeset: latexmk (uplatex -> dvipdfmx)
\documentclass[uplatex,dvipdfmx]{vkaishi}

%% vuccaken.sty などを読み込むには,環境変数 `TEXINPUTS` に
%% このファイルからの相対パス `../../sty/` を追加する必要があります.
\usepackage{vuccaken}
% \usepackage{v-hyperref}

%% スタイルファイルの読み込み(\usepackage{})や自作マクロは,
%% プリアンブル(ここ)ではなく,自分のスタイルファイル(.sty)に書く.
\usepackage{template}

\begin{document}% - - - - - - - 以下本文 - - - - - - - - - -

\mokuji{2} % 目次出力

%% - - - - - - - - - - - - - - - - - - - - - - - - %%
\kaishititle%
  {会誌原稿テンプレート}% title
  {テンプレ科学科4回生}% 所属
  {\vname{テンプレ}{太郎}}% name
%% - - - - - - - - - - - - - - - - - - - - - - - - %%


%% - - - - - - - - - - - - - - - - - - - - - - - -
\section*{はじめに}
%% - - - - - - - - - - - - - - - - - - - - - - - -
はじめに背景・導入・序論などを書く.
概要を書いてもよい.


%% - - - - - - - - - - - - - - - - - - - - - - - -
\section{セクション}
%% - - - - - - - - - - - - - - - - - - - - - - - -
chapterは使わない\footnote{kaishi titleがchapterに相当します}.
section以下は自由に使ってよし.


%% - - - - - - - - - - - - - - - - - - - - - - - -
%%  参考文献
%% - - - - - - - - - - - - - - - - - - - - - - - -
%% \bibitem のlabelはユニークでなければいけないので,各自変更すること.
%% 著者名+出版年でラベリングするのがおすすめです.
\begin{thebibliography}{99}
  \bibitem{label-1} 著者1・著者2,『本のタイトル』,出版社,出版年.
  \bibitem{label-2} ページの著者,『ページのタイトル』,最終アクセス日,\\(\url{https://vuccaken.github.io}).
\end{thebibliography}

\end{document} % - - - - - - - - - - - - - - - - - - - - -
%%
%% ファイトだよ!
%%