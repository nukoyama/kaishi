%
%     奥付
%

\documentclass[10pt,b5paper,papersize,dvipdfmx]{jsbook}

\usepackage{vuccaken}
\usepackage{vuccaken2019}

% --------------------------------------
\begin{document}

% \thispagestyle{empty}

\markboth{}{} % headerのchaper nameとsection nameを消す
\clearpage % 右開きじゃなくて良い

\noindent%
{\headfont\large 物理科学研究会の略歴}
\par\vspace{.5zw}%

\begin{tabular}{ll}
  1949年 & 核物理研究会として発足 \\
  1973年 & 現存する最古の会誌 出版 \\
  2000年 & 物理科学研究会に改名   \\
  2016年 & 会誌『白夜 第一号』出版 \\
  2017年 & OB会の開催           \\
        & 会誌『白夜 第二号』出版 \\
  2018年 & 会誌『白夜 第三号』出版 \\
  2019年 & 会誌『白夜 第四号』出版 \\
        & OB会の開催(予定)
\end{tabular}


\vfill

\noindent%
\hspace{2zw}{\headfont 令和元年度 物理科学研究会誌}
\par\noindent%
\hspace{2zw}{\headfont\large 白夜 第四号}

\vspace{\baselineskip}\vspace{-1zw}
\hrule\hrule

{ %% \setlength の効果範囲
  \setlength{\tabcolsep}{0em} % tabular horizontal padding

  \newcolumntype{P}{>{\hspace{2zw}}r}
  \newcolumntype{R}{>{\:}r}
  \newcolumntype{L}{>{\hspace{2zw}}l}
  \noindent
  \begin{tabularx}{\textwidth}{PRRL}
    2019年 & 12月 & 1日 & 初版発行 \\
    % 2020年 & 1月 & 1日 & 第2版発行
  \end{tabularx}

  \vspace{1zw}\noindent
  \begin{tabularx}{\textwidth}{Pl}
    {\headfont 表紙イラスト}: & \fjwr \\
    {\headfont 編集・校正}: & \nkym \\
    {\headfont 発行所}: & 立命館大学 物理科学研究会 \\
      & 〒525-8577 滋賀県 草津市 野路東 1-1-1 \\
      & 立命館大学 BKC アクトα サークルルーム6 \\
      & メール: \url{vuccaken@gmail.com} \\
      & ホームページ: \url{vuccaken.github.io} \\
      & Twitter: \url{@vuccaken}
  \end{tabularx}
}

\hrule\hrule

%% - - - - - - - - - - -

\end{document}

%
% お疲れさまです
%