%%
%% 光子の偏光軸

\documentclass[border=3pt,tikz,dvipdfmx,10pt]{standalone}
% \documentclass[dvipdfmx,10pt]{jsbook}

\usepackage{tikz}
% \usetikzlibrary{quotes,angles,arrows}
\usetikzlibrary{decorations.pathmorphing}
\usetikzlibrary{calc}
%
\begin{document}
%

\begin{tikzpicture}
  [
    x=30mm, y=30mm,
    line cap=round, line join=round,
    axis/.style={black, >=latex, ->},
  ]
\small

\def\X{2}
\def\CT{1.4}
\coordinate (O) at (0,0);
\draw[axis] (O) ++(-.1,0) -- ++(1.1*\X,0) node[right] {$x$};
\draw[axis] (O) ++(0,-.1) -- ++(0,.9*\CT) node[above] {$ct$};

% \coordinate (A) at (.3*\X,.2*\X);
% \node[below] at (A) {A};
% \def\tmax{\X}
% \draw (A)++(-\tmax,\tmax) -- (A) -- ++(\tmax,\tmax);
\def\eventCT{.15*\CT}
\def\tmax{.5*\CT}

\newcommand\lightCone[2]{
  \coordinate (#1) at (#2,\eventCT);
  \node[below] at (#1) {#1};
  \draw[thick] (#1)++(-\tmax,\tmax) -- (#1) -- ++(\tmax,\tmax);
}
\def\centerX{.5*\X}
\def\aaaa{.25*\CT}
\lightCone{A}{\centerX -\aaaa}
\lightCone{B}{\centerX +\aaaa}

\def\intscCT{\eventCT+\aaaa}
\node[above=15pt] at (\centerX,\intscCT) {共通未来};
\draw[dashed] (-.5,\eventCT) -- (\X,\eventCT);
\draw[dashed] (-.5,\intscCT) -- (\X,\intscCT);

\draw[<->,>=stealth] (-.1,\eventCT) -- (-.1,\intscCT);
% \def\centerCT{.5*\intscCT + .5*\eventCT}
\def\centerCT{11.5mm}
\node[left=10pt, text width=10mm,align=center] at (0,\centerCT) {AとBは独立};

\end{tikzpicture}

%
%
\end{document}