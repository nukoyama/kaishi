%%
%%   N  K  Y  M
%%    A  A  A  A
%%

%% typeset: latexmk (uplatex -> dvipdfmx)
\documentclass[uplatex,dvipdfmx]{vkaishi}

% \usepackage{v-hyperref}
\usepackage{vuccaken}
\usepackage{nkym}

% \analogtrue
% \colorfalse

\begin{document}% - - - - - - - 以下本文 - - - - - - - - - -

\mokuji{2} % 目次出力

%% - - - - - - - - - - - - - - - - - - - - - - - - %%
\kaishititle%
  {ブラックホールのエントロピー}% タイトル
  {理工研究科物理科学コースM1回生}% 所属・回生
  {\vname{中須}{かすみ}}% 名前
%% - - - - - - - - - - - - - - - - - - - - - - - - %%


%% - - - - - - - - - - - - - - - - - - - - - - - -
\section*{はじめに}
%% - - - - - - - - - - - - - - - - - - - - - - - -
こんにちわ


%% - - - - - - - - - - - - - - - - - - - - - - - -
\section{セクション}
%% - - - - - - - - - - - - - - - - - - - - - - - -
ああああ


%% - - - - - - - - - - - - - - - - - - - - - - - -
%%  参考文献
%% - - - - - - - - - - - - - - - - - - - - - - - -
\begin{thebibliography}{99}
  \bibitem{label-1} 著者1・著者2,『本のタイトル』,出版社,出版年.
  \bibitem{label-2} ページの著者,『ページのタイトル』,最終アクセス日,\\(\url{https://vuccaken.github.io}).
\end{thebibliography}

\end{document} % - - - - - - - - - - - - - - - - - - - - -
%%
%% ファイトだよ!
%%